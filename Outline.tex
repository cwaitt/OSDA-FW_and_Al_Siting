% Created 2021-07-08 Thu 11:10
% Intended LaTeX compiler: pdflatex
\documentclass[journal=accacs,manuscript=article, email=true, layout=traditional]{achemso}
  \setkeys{acs}{biblabel=brackets,super=true,articletitle=False,maxauthors=0}
\usepackage[utf8]{inputenc}
\usepackage[T1]{fontenc}
\usepackage{fixltx2e}
\usepackage{url}
\usepackage{siunitx}
\usepackage{mhchem}
\usepackage{graphicx}
\usepackage{color}
\usepackage{amsmath}
\usepackage{textcomp}
\usepackage{wasysym}
\usepackage{latexsym}
\usepackage{amssymb}
\usepackage{minted}
\usepackage[section]{placeins}
\usepackage[linktocpage, pdfstartview=FitH, colorlinks=true, linkcolor=blue, anchorcolor=blue, citecolor=blue, filecolor=blue, menucolor=blue, urlcolor=blue]{hyperref}
\usepackage{attachfile}
\newcommand{\red}[1]{\textcolor{red}{#1}}
\newcommand{\blue}[1]{\textcolor{blue}{#1}}
\keywords{}
\renewcommand{\thefigure}{\arabic{figure}}
\renewcommand{\thetable}{\arabic{table}}
\usepackage{lmodern}
\usepackage{cleveref}
\author{Craig Waitt}
\affiliation{Department of Chemistry and Biochemistry, University of Notre Dame, Notre Dame, Indiana 46556, United States}
\author{Xuyao Gao}
\affiliation{Department of Chemical and Biomolecular Engineering, University of Notre Dame, Notre Dame, Indiana 4656, United States}
\author{William F. Schneider}
\email{wschneider@nd.edu}
\affiliation{Department of Chemical and Biomolecular Engineering, University of Notre Dame, Notre Dame, Indiana 46556, United States}
\alsoaffiliation{Department of Chemistry and Biochemistry, University of Notre Dame, Notre Dame, Indiana 46556, United States}
\date{}
\title{Developing Relationships between FW/OSDA orientation and Al siting}
\begin{document}

\begin{header}
\end{header}

\newpage

\section{Workflow}
\label{sec:orge74f7bd}

\subsection{Objectives}
\label{sec:org65c4bf6}

\begin{enumerate}
\item Develop models to capture/describe the orientational preference a framework (FW) has for a particular organic structure directing agent (OSDA)
\item Identify FW/OSDA combinations that could influence Al ordering through preferential OSDA orientations.
\item Compute and compare organic and inorganic structure directing agent interactions with frameworks to identify consequences on Si/Al ratios
\end{enumerate}

\subsection{Plan of Attack}
\label{sec:org1266e79}

\begin{enumerate}
\item \textbf{What is the preferential orientational of a specific OSDA/Zeolite combination}

\begin{enumerate}
\item Develop a model to capture OSDA Orientations
\begin{itemize}
\item Classical molecular dynamics (CMD) at very high temperatures (\(T \geq 5,000\)) to sample different possible orientation
\end{itemize}

\item Compute interaction energy (IE) of all FW and OSDA combinations (maybe free energies):
\begin{itemize}
\item CHA and AEI with TMADA, DMDMP (isomers), and DEDMP (isomers)
\begin{itemize}
\item include LTA and TMA?
\end{itemize}
\item Using CMD at \SI{343}{K} (rigid framework)
\item Using density functional theory with dispersion (DFT-D3) (flexible framework)
\begin{itemize}
\item Rigid Framework as well?
\end{itemize}
\end{itemize}

\item Evaluate orientation and positioning dependence of OSDA in FW
\begin{itemize}
\item Measure the orientation of an OSDA relative to some axis which encompases the cage the OSDA is in
\item Measure the location of the OSDA center of mass is relative to the center of the zeolite cage
\end{itemize}
\end{enumerate}

\item \textbf{What is the impact of Al siting with respect to OSDA orientation}

\begin{enumerate}
\item Using some preferred orientation(s)
\begin{itemize}
\item Using the same technique developed by Sichi, run CMD on Al substituted FW and charged OSDA as in 1(a)
\item Refine using techniques in 1(b)
\end{itemize}
\item Compare relative energies (or IE?)
\begin{itemize}
\item Is there a difference between AEI and CHA?
\end{itemize}
\end{enumerate}

\item \textbf{What is the impact of co-caging (Relevent?)}

\begin{enumerate}
\item Charged or chargless model
\end{enumerate}
\end{enumerate}
\end{document}